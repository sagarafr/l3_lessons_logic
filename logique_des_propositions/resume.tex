\documentclass[11pt,twoside,openright,a4paper]{report}

\usepackage{tocloft}
\usepackage{graphicx}
\usepackage[utf8]{inputenc}
\usepackage[francais]{babel}
\selectlanguage{francais}

\oddsidemargin 0.8cm
\evensidemargin -0.2cm
\textwidth 15cm
\textheight 23cm
\topmargin -0.5cm

\renewcommand{\familydefault}{\sfdefault}

% table of contents config

\tocloftpagestyle{empty}
\setcounter{tocdepth}{1}

%

%\usepackage{sectsty}
\usepackage{fancyhdr}

\renewcommand{\chaptermark}[1]{\markboth{\thechapter\quad #1}{}}
\renewcommand{\sectionmark}[1]{\markright{#1\quad \thesection}}
\lhead[\fancyplain{}{\raggedright\slshape\leftmark}]{}
\chead{}
\rhead[]{\fancyplain{}{\raggedleft\slshape\rightmark}}
\lfoot[\fancyplain{}{\thepage}]{}
\cfoot{}
\rfoot[]{\fancyplain{\thepage}{\thepage}}

%\setlength{\headrulewidth}{0.4pt}
%\setlength{\footrulewidth}{0.4pt}
%\setlength{\plainfootrulewidth}{0.4pt}

\usepackage{listings}
\usepackage{xcolor}

\usepackage{pgf}
\usepackage{tikz}
\usetikzlibrary{arrows,automata}
% \usepackage[latin1]{inputenc}

\begin{document}

\pagestyle{fancy}

\chapter{Logique des propositions} % (fold)
\label{cha:logique_des_propositions}

\paragraph{Objectif} % (fold)
\label{par:objectif}

Donner un fondement formel du raisonnement dans un ensemble restreint d’énoncés

% paragraph objectif (end)

\paragraph{Syntaxe} % (fold)
\label{par:syntaxe}

\begin{center}
	\begin{tabular}{ | l | l | }
		\hline
		Connecteur & Nom de formule \\
		\hline
		$\neg$ & Négation \\
		\hline
		$\land$ & Et \\
		\hline
		$\lor$ & Ou \\
		\hline
		$\Rightarrow$ & Implication \\
		\hline
		$\Leftrightarrow$ & Équivalence \\
		\hline
		$\bot$ & Faux \\
		\hline
		$\top$ & Vrai \\
		\hline
	\end{tabular}
\end{center}

% paragraph syntaxe (end)

\paragraph{Formules atomiques} % (fold)
\label{par:formules_atomiques}

Une formule atomique est composé de : $\bot$, $\top$ et de variables propositionnelles.

% paragraph formules_atomiques (end)

\paragraph{Formules} % (fold)
\label{par:formules}

\begin{itemize}
	\item Les formules atomiques sont des formules.
	\item Si $A$ est une formule, alors $\neg A$ est une formule.
	\item Si $A$ et $B$ sont des formules, alors $(A \land B), (A \lor B), (A \Rightarrow B) et (A \Leftrightarrow B)$ sont des formules.		
\end{itemize}

% paragraph formules (end)

\paragraph{Sous formules} % (fold)
\label{par:sous_formules}

Chaque formule $F$ a un ensemble de sous-formules $sf(F)$ défini par :

\begin{itemize}
	\item Si $F$ est une variable propositionnelle alors $sf(F) = \{F\}$.
	\item Si $F$ est la formule $\neg G$ alors $sf(F) = sf(G) \bigcup \{F\}$.
	\item Si $F$ est la formule $(G \land H)$, $(G \lor H)$, $(G \Rightarrow H)$ et $(G \Leftrightarrow H)$, alors $sf(F) = sf(G) \bigcup sf(H) \bigcup \{F\}$.
\end{itemize}

% paragraph sous_formules (end)

\paragraph{Exemple} % (fold)
\label{par:exemple}
Les sous formules de $(( p \Rightarrow q ) \land p ) \Rightarrow \neg q$ sont:

\begin{itemize}
	\item $p$
	\item $q$
	\item $(p \Rightarrow q)$
	\item $\neg q$
	\item $( p \Rightarrow q ) \land p$
	\item $(( p \Rightarrow q ) \land p ) \Rightarrow \neg q$
\end{itemize}

% paragraph exemple (end)

\paragraph{Substitution} % (fold)
\label{par:substitution}

Soit $F$ une formule, soit $G$ une sous-formule de $F$, soit $H$ une formule. $F_{H\backslash G}$ est obtenu en substituant $H$ à chaque occurrence de $G$ dans $F$. $F_{H\backslash G}$ est une formule.

% paragraph substitution (end)

\paragraph{Exemple} % (fold)
\label{par:exemple}

Soit $F$ la formule $(( p \Rightarrow q ) \land p ) \Rightarrow \neg q$. Soit $G$ la sous-formule $(p \Rightarrow q)$. Soit $H$ la formule $(s \land t)$. $F_{H \backslash G}$ est la formule $(( s \land t ) \land p ) \Rightarrow \neg q$

% paragraph exemple (end)

% chapter logique_des_propositions (end)

\end{document}