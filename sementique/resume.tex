\documentclass[11pt,twoside,openright,a4paper]{report}

\usepackage{tocloft}
\usepackage{graphicx}
\usepackage[utf8]{inputenc}
\usepackage[francais]{babel}
\selectlanguage{francais}

\oddsidemargin 0.8cm
\evensidemargin -0.2cm
\textwidth 15cm
\textheight 23cm
\topmargin -0.5cm

\renewcommand{\familydefault}{\sfdefault}

% table of contents config

\tocloftpagestyle{empty}
\setcounter{tocdepth}{1}

%

%\usepackage{sectsty}
\usepackage{fancyhdr}

\renewcommand{\chaptermark}[1]{\markboth{\thechapter\quad #1}{}}
\renewcommand{\sectionmark}[1]{\markright{#1\quad \thesection}}
\lhead[\fancyplain{}{\raggedright\slshape\leftmark}]{}
\chead{}
\rhead[]{\fancyplain{}{\raggedleft\slshape\rightmark}}
\lfoot[\fancyplain{}{\thepage}]{}
\cfoot{}
\rfoot[]{\fancyplain{\thepage}{\thepage}}

%\setlength{\headrulewidth}{0.4pt}
%\setlength{\footrulewidth}{0.4pt}
%\setlength{\plainfootrulewidth}{0.4pt}

\usepackage{listings}
\usepackage{xcolor}

\usepackage{pgf}
\usepackage{tikz}
\usetikzlibrary{arrows,automata}
\usepackage{amssymb}

\begin{document}

\pagestyle{fancy}

\chapter{Sémentique} % (fold)
\label{cha:s_mentique}

\paragraph{Intérpretation} % (fold)
\label{par:int_rpretation}

Une interpretation I (ou valuation) est une application de l'ensemble des variables propsitionnelles dans l'ensemble des valeurs de vérité $\{0,1\}$.

% paragraph int_rpretation (end)

\paragraph{Exemple} % (fold)
\label{par:exemple}

Soit un ensemble $P=\{p_1, p_2, \ldots\}$, un ensemble de variables propositionnelles. L'application $I$ telle que $I(p_1)=I(p_2)=\ldots=1$ est une intérpretation.

% paragraph exemple (end)

\paragraph{Tables de vérité} % (fold)
\label{par:tables_de_v_rit_}

Représentation simple de l'ensemble des intérprétations possibles sur une formule donnée $F$.

% paragraph tables_de_v_rit_ (end)

\paragraph{Exemple} % (fold)
\label{par:exemple}

\begin{center}

	\begin{tabular}{|c|c|c|c|c|c|}
		\hline
		$p$ & $q$ & $p \land q$ & $p \lor q$ & $p \Rightarrow q$ & $p \Leftrightarrow q$ \\
		\hline
		$1$ & $1$ & $1$ & $1$ & $1$ & $1$ \\
		$1$ & $0$ & $0$ & $1$ & $0$ & $0$ \\
		$0$ & $1$ & $0$ & $1$ & $1$ & $0$ \\
		$0$ & $0$ & $0$ & $0$ & $1$ & $1$ \\
		\hline
	\end{tabular}

\end{center}

\begin{center}

	\begin{tabular}{|c|c|}
		\hline
		$p$ & $\neg p$ \\
		\hline
		$1$ & $0$ \\
		$0$ & $1$ \\
		\hline
	\end{tabular}
	
\end{center}

% paragraph exemple (end)

\paragraph{Formules équivalentes} % (fold)
\label{par:formules_quivalentes}

2 formules sont équivalentes si elles ont la même table de vérité. Pour montrer que 2 formules sont équivalentes, on peut passer par des équivalences remarquables.

% paragraph formules_quivalentes (end)

\paragraph{Équivalences remarquables} % (fold)
\label{par:_quivalences_remarquables}

\begin{itemize}
	\item Idempotence :\\
	
	\begin{center}

		\begin{tabular}{lr}
			$F \lor F \equiv F$ & $F \land F \equiv F$ \\
			$F \lor \top \equiv \top$ & $F \land \bot \equiv \bot$\\
		\end{tabular}
	
	\end{center}

	\item Complémentarité :\\

	\begin{center}

		\begin{tabular}{lr}
			$F \lor \neg F \equiv \top$ & $F \land \neg F \equiv \bot$ \\
		\end{tabular}
	
	\end{center}

	\item Élément neutre :\\

	\begin{center}

		\begin{tabular}{lr}
			$F \lor \bot \equiv F$ & $F \land \top \equiv F$ \\
		\end{tabular}
	
	\end{center}

	\item Involution :\\

	\begin{center}

		\begin{tabular}{c}
			$\neg \neg F \lor F \equiv F$ \\
		\end{tabular}
	
	\end{center}

	\item Commutativité :\\

	\begin{center}

		\begin{tabular}{lr}
			$F \lor G \equiv G \lor F$ & $F \land G \equiv G \land F$ \\
		\end{tabular}
	
	\end{center}

	\item Associativité :\\

	\begin{center}

		\begin{tabular}{c}
			$F \lor (G \lor H) \equiv (F \lor G) \lor H$ \\
			$F \land (G \land H) \equiv (F \land G) \land H$ \\
		\end{tabular}
	
	\end{center}

	\item Distributivité :\\

	\begin{center}

		\begin{tabular}{c}
			$F \land (G \lor H) \equiv (F \land G) \lor (F \land H)$ \\
			$F \lor (G \land H) \equiv (F \lor G) \land (F \lor H)$ \\
		\end{tabular}
	
	\end{center}

	\item Absorption :\\

	\begin{center}

		\begin{tabular}{ll}
			$F \land (F \lor G) \equiv F$ & $F \lor (F \land G) \equiv F$\\
			$F \land (\neg F \lor G) \equiv F \land G$ & $F \lor (\neg F \land G) \equiv F \lor G$\\
		\end{tabular}
	
	\end{center}

	\item Loi de De Morgan :\\

	\begin{center}

		\begin{tabular}{lr}
			$\neg (F \land G) \equiv \neg F \lor \neg G$ & $\neg (F \lor G) \equiv \neg F \land \neg G$\\
		\end{tabular}
	
	\end{center}

	\item Autres :\\

	\begin{enumerate}
		\item $F \Rightarrow \bot \equiv \neg F$
		\item $(F \Leftrightarrow G) \equiv (F \Rightarrow G) \land (G \Rightarrow F)$
		\item $F \Rightarrow G \equiv \neg F \lor G \equiv \neg G \Rightarrow \neg F$
		\item $F \Rightarrow (G \Rightarrow H) \equiv F \land G \Rightarrow H$
	\end{enumerate}

\end{itemize}

% paragraph _quivalences_remarquables (end)

\paragraph{Modèle} % (fold)
\label{par:mod_le}

Soit $I$ une intérprétation, $F$ une formule et $\Gamma = \{F_1, F_2, \ldots, F_n\}$, un ensemble de formules. $I$ est un modèle de $F$, noté $I \vDash F$, si $I(F) = 1$. $I$ est un modèle de $\Gamma$, si $I$ est un modèle de toutes formules de $F$ de l'ensemble $\Gamma$.

% paragraph mod_le (end)

\paragraph{Exemple} % (fold)
\label{par:exemple}

\begin{itemize}
	\item Soit $F$ la formule $p \Rightarrow q$. L'intérprétation $I$ telle que $I(p) = 1$ et $I(q) = 1$ est un modèle de $F$.
	\item Soit $\Gamma = \{p \Rightarrow q, q \Rightarrow p, \top\}$. L'intérprétation $I$ telle que $I(p) = 1$ et $I(q) = 1$ est un modèle de $\Gamma$ car $I(p \Rightarrow q) = I(q \Rightarrow p) = I(\top) = 1$.
\end{itemize}



% paragraph exemple (end)

% chapter s_mentique (end)

\end{document}
